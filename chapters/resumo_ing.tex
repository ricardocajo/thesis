% !TeX root = ../main.tex
% %%%%%%%%%%%%%%%%%%%%%%%%%%%%%%%%%%%%%%%%%%%%%%%%%%%%%%%%%%%%%%%%%%%%%%%%%%%%%%
% Abstract in English
\vspace*{2cm}
\begin{center}
\Large \bf Abstract
\end{center}
\vspace*{1cm} \setlength{\baselineskip}{0.6cm}

Robotics has a big influence in today's society, so much that a potential failure in a robot may have extraordinary costs, not only financial but can also cost lives.

Current practices in robot testing are vast and involve such methods as simulations, log checking, or field tests. The frequent common denominator between these practices is the need for human visualization to determine the correctness of a given behavior. Automating this analysis could not only relieve this burden from a high-skilled engineer but also allow for massive parallel executions of tests, that could potentially detect behavioral faults in the robotic system that would otherwise not be found due to human error or lack of time.

For my thesis, I have developed a domain-specific language to specify the properties of robotic systems in ROS. Specifications written by developers in this language can be compiled to a monitor ROS module, that will detect violations of those properties. I have used this language to express the temporal and positional properties of robots, and we have automated the monitoring of some behavioral violations of robots in relation to their state or events during a simulation.

*Evaluation ?*

\vfill

\begin{flushleft}
\textbf{Keywords:}
Robotics, Domain-specific language, Runtime Monitoring, Error detection, Automation
\end{flushleft}
