% !TeX root = ../main.tex
% %%%%%%%%%%%%%%%%%%%%%%%%%%%%%%%%%%%%%%%%%%%%%%%%%%%%%%%%%%%%%%%%%%%%%%%%%%%%%%
% Abstract in English
\vspace*{2cm}
\begin{center}
\Large \bf Abstract
\end{center}
\vspace*{1cm} \setlength{\baselineskip}{0.6cm}

Robotic systems are critical in today's society, a potential failure in a robot may have extraordinary costs, not only financial but can also cost lives.

Current practices in robot testing are vast and involve methods like simulation, log checking, or field testing. However current practices often require human monitoring to determine the correctness of a given behavior. Automating this analysis can not only relieve the burden from a high-skilled engineer but also allow for massive parallel executions of tests, that can detect behavioral faults in the robotic system that would otherwise not be found due to human error or lack of time.

For this work, we have developed a domain-specific language to specify the properties of robotic systems in the Robot Operating System (ROS). Specifications written by developers in this language are compiled to a monitor ROS module, that detects violations of those properties in runtime. We have used this language to express the temporal and positional properties of robots, and we have automated the monitoring of some behavioral violations of robots in relation to their state or events during a simulation.

*Evaluation ?*

\vfill

\begin{flushleft}
\textbf{Keywords:}
Robotics, Domain-specific language, Runtime Monitoring, Error detection
\end{flushleft}
