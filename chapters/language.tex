% !TeX root = ../main.tex
% %%%%%%%%%%%%%%%%%%%%%%%%%%%%%%%%%%%%%%%%%%%%%%%%%%%%%%%%%%%%%%%%%%%%%%%%%%%%%%
% Language
\chapter{Specification Language for Robotics Properties}
\label{chap:language}

In this chapter, the structure and intricacies of the DSL are presented. *Explain each section*

% ------------------------------------------------------------------------------
% High-Level Notations
\section{High Level Notations}

\begin{itemize}
\item \textbf{Property -} A property represents a temporal specification or a blend of temporal specifications between components.
\item \textbf{Declaration -} A declaration allows for the representation of ROS \textit{topics} in order to interact with it.
\item \textbf{Model -} A model allows for the declaration of specific \textit{topics} that are required when correlating certain robots' and simulation components.
\item \textbf{Association -} An association serves as a way to create program variables.
\end{itemize}

% ------------------------------------------------------------------------------
% Temporal Keywords
\subsection{Temporal Keywords}

We consider not only LTL basic operators but also some common shortcuts for useful combinations of such operators, like \textit{after\_until}.

\begin{itemize}
\item {\bfseries always X} - X has to hold on the entire subsequent path;
\item {\bfseries never X} - X never holds on the entire subsequent path;
\item {\bfseries eventually X} - X eventually has to hold somewhere on the subsequent path;
\item {\bfseries after X, Y} - after the event X is observed, Y has to hold on the entire subsequent path;
\item {\bfseries until X, Y} - X holds at the current or future position, and Y has to hold until that position. At that position, Y does not have to hold anymore;
\item {\bfseries after\_until X, Y, Z} - after the event X is observed, Z has to hold on the entire subsequent path up until Y happens. At that position, Z does not have to hold anymore;
\end{itemize}

%\noindent It is also possible to reference previous states of variables, using \textbf{\lstinline|@{X, -y}|}, representing the value of variable \lstinline|X| at time \lstinline|-y|.

% ------------------------------------------------------------------------------
% Temporal value
\subsection{Temporal value}

It is also possible to reference previous variable states:
\begin{equation}
@\{X, -y\}
\end{equation}
This will represent the value of the variable X in the point in time -y.

% ------------------------------------------------------------------------------
% Functions
\subsection{Simulation primivitives}

To support comparing the internal state of the robotic system with the environment, we provide basic primitives in the language to refer to the simulation environment:

\begin{itemize}
\item {\bfseries X.position} - The position of the robot in the simulation;
\item {\bfseries X.position.y} - The position in the y axis of the robot in the simulation. Also works for x and z;
\item {\bfseries X.distance.Y} - The absolute distance between two objects in the simulation. For the x and y axis;
\item {\bfseries X.distanceZ.Y} - The absolute distance between two objects in the simulation. For the x, y, and z axis;
\item {\bfseries X.velocity} - The velocity of an object in the simulation. This refers to linear velocity;
\item {\bfseries X.velocity.x} - The velocity in the x axis of an object in the simulation. This refers to linear velocity;
\item {\bfseries X.localization\_error} - The difference between the robot's perception of its position and the actual position in the simulation;
\end{itemize}

% ------------------------------------------------------------------------------
% Operands
\section{Operands}

Besides the already mentioned operands, \textit{Temporal values}, \textit{Simulation primivitives}, and \textit{Temporal Keywords}, the DSL also supports both Integer and Float values, Booleans, and declared variables.

% ------------------------------------------------------------------------------
% Operators
\section{Operators}

The DSL supports operators to correlate components. The operators are \textit{+} (addition), \textit{-} (subtraction), \textit{*} (multiplication), \textit{/} (division), \textit{==} (equals), \textit{!=} (different), \textit{>} (greater than), \textit{>=} (greater or equal than), \textit{<} (lower than), \textit{<=} (lower or equal than), \textit{and} (conjunction), \textit{or} (disjunction), \textit{implies} (implication), and for any comparison operator X \textit{X{y}} - the values being compared will have an error margin of y (Example: Z =={0.05} Y).

% ------------------------------------------------------------------------------
% Protected Variables
\section{Protected Variables}

Protected variables are variable names restricted to set determined monitoring parameters.

\_rate\_ - Set the frame rate which properties are checked (By default, the rate is 30hz)

\_timeout\_ - Set the timeout for how long the verification will last (By default, the timeout is 100 seconds)

\_margin\_ - Set the error margin for comparisons

% ------------------------------------------------------------------------------
% Topic declaration
\section{Topic declaration}

In order to relate robot components with the simulation, the developer can declare the relevant \textit{topics}.

The language cannot inherently have a way to interact with specific components of a robot because it does not know which topic to get information from. Therefore, the developer needs to declare these specific topics to be able to interact with them.

\textit{The variable robot\_position was declared with the type Odometry.pose.pose.position and is linked to the topic /odom;}

%\vspace{2mm}

\texttt{decl robot\_position /odom Odometry.pose.pose.position}

% ------------------------------------------------------------------------------
% Model robots
\section{Model robots}

A set of specific topics can be modeled for the robot, like \textit{position} or \textit{velocity}. The compiler will use these to call specific functions that need this information from the robot's perspective.

%\vspace{2mm}

\texttt{model robot1:}

\texttt{    position /odom Odometry.pose.pose.position}

\texttt{    ;}

%\vspace{2mm}

\texttt{never robot1.localization error > 0.002}

% ------------------------------------------------------------------------------
% Grammar
\section{Grammar}

\begin{bnfgrammar}
    <program> : Start
    ::=
    <command> || <command> <program>
    ;;
    <command> ::=
    <association>
    | <declaration>
    | <model>
    | <pattern>
    ;; 
    <association> ::=
    name = <pattern>
    | \_rate\_ = integer
    | \_timeout\_ = <number>
    | \_default\_margin\_ = <number>
    ;;
    <declaration> ::=
    decl name topic\_name <msgtype>
    | decl name name <msgtype>
    ;;
    <model> ::= 
    model name \: <modelargs> ; %TODO escape :
    ;;
    <modelargs> ::= 
    <name> topic\_name <msgtype>
    | <name> <name> <msgtype>
    | <name> topic\_name <msgtype> <modelargs>
    | <name> <name> <msgtype> <modelargs>
    ;;
    <name> ::= 
    name || <func\_main>
    ;;
    <func\_main> ::= 
    position
    | velocity
    | distance
    | localization\_error
    | orientation
    ;;
    <msgtype> ::= 
    <name> || <name> . <msgtype>
    ;;
    <pattern> ::= 
    always <pattern>
    | never <pattern>
    | eventually <pattern>
    | after <pattern> , <pattern>
    | until <pattern> , <pattern>
    | after\_until <pattern> , <pattern> , <pattern>
    | <conjunction>
    ;;
    <conjunction> ::= 
    <conjunction> and <comparison>
    | <conjunction> or <comparison>
    | <conjunction> implies <comparison>
    | <comparison>
    ;;
    <comparison> ::= 
    <multiplication> <opbin> <multiplication>
    | <multiplication> <opbin> { <number> } <multiplication>
    | <multiplication>
    ;;
    <opbin> ::= 
    < || > || <= || >= || == || !=
    ;;
    <multiplication> ::= 
    <multiplication> * <addition>
    | <multiplication> / <addition>
    | <addition>
    ;;
    <addition> ::= <addition> + <operand>
    | <addition> - <operand>
    | <operand>
    ;;
    <operand> ::= 
    name || <number> || true || false || <func> || <temporalvalue> || ( <pattern> )
    ;;
    <number> ::= 
    float || integer
    ;;
    <func> ::= 
    name . <func\_main>
    | name . <func\_main> <funcargs>
    ;;
    <funcargs> ::= 
    . <name> || . <name> <funcargs>
    ;;
    <temporalvalue> ::= 
    @ { name , integer }
\end{bnfgrammar}
