% !TeX root = ../main.tex
% %%%%%%%%%%%%%%%%%%%%%%%%%%%%%%%%%%%%%%%%%%%%%%%%%%%%%%%%%%%%%%%%%%%%%%%%%%%%%%
% Language
\chapter{Language}
\label{chap:language}

% ------------------------------------------------------------------------------
% High Level Notations
\section{High Level Notations}

\begin{itemize}
\item \textbf{Declaration -} aa
\item \textbf{Property -} aa
\item \textbf{Model -} aa
\item \textbf{Association -} aa
\end{itemize}

% ------------------------------------------------------------------------------
% Properties
\subsection{Properties}

\begin{itemize}
\item[--] always X (X has to hold on the entire subsequent path);
\item[--] never X (X never holds on the entire subsequent path);
\item[--] eventually X (X eventually has to hold, somewhere on the subsequent path);
\item[--] after X, Y (after the event X is observed, Y has to hold on the entire subsequent path);
\item[--] until X, Y (X holds at the current or future position, and Y has to hold until that position. At that position, Y does not have to hold anymore);
\item[--] after\_until X, Y, Z (after the event X is observed, Z has to hold on the entire subsequent path up until Y happens, at that position Z does not have to hold anymore);
\end{itemize}

% ------------------------------------------------------------------------------
% Operands
\section{Operands}

Besides Number / Boolean / Var

\begin{itemize}
\item \textbf{Temporal value -} aa
\item \textbf{Function -} aa
\item \textbf{Property -} aa
\end{itemize}

% ------------------------------------------------------------------------------
% Temporal value
\subsection{Temporal value}

It is also possible to reference previous variable states:
\begin{equation}
@\{X, -y\}
\end{equation}
This will represent the value of the variable X in the point in time -y.

% ------------------------------------------------------------------------------
% Functions
\subsection{Functions}

\begin{itemize}
\item[--] X.position (The position of the robot in the simulation);
\item[--] X.position.y (The position in the y axis of the robot in the simulation. Also works for x and z);
\item[--] X.distance.Y (The absolute distance between two objects in the simulation. For the x and y axis);
\item[--] X.distanceZ.Y (The absolute distance between two objects in the simulation. For the x, y, and z axis);
\item[--] X.velocity (The velocity of an object in the simulation. This refers to linear velocity);
\item[--] X.velocity.x (The velocity in the x axis of an object in the simulation. This refers to linear velocity);
\item[--] X.localization\_error - The difference between the robot's perception of its position and the actual position in the simulation;
\end{itemize}

% ------------------------------------------------------------------------------
% Protected Variables
\section{Protected Variables}

\_rate\_ - Set the frame rate which properties are checked (By default the rate is 30hz)

\_timeout\_ - Set the timeout for how long the verification will last (By default the timeout is 100 seconds)

\_margin\_ - Set the error margin for comparisons

% ------------------------------------------------------------------------------
% Grammar
\section{Grammar}

%\begin{bnfgrammar}
%    term1 ::= rhs1
%    ;;
%\end{bnfgrammar}
