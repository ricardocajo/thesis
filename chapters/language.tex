% !TeX root = ../main.tex
% %%%%%%%%%%%%%%%%%%%%%%%%%%%%%%%%%%%%%%%%%%%%%%%%%%%%%%%%%%%%%%%%%%%%%%%%%%%%%%
% Language
\chapter{Specification Language for Robotics Properties}
\label{chap:language}

*structure explanation*

% ------------------------------------------------------------------------------
% High Level Notations
\section{High Level Notations}

\begin{itemize}
\item \textbf{Declaration -} aa
\item \textbf{Property -} aa
\item \textbf{Model -} aa
\item \textbf{Association -} aa
\end{itemize}

% ------------------------------------------------------------------------------
% Properties
\subsection{Temporal Keywords}

\begin{itemize}
\item[--] {\bfseries always X} - X has to hold on the entire subsequent path;
\item {\bfseries never X} - X never holds on the entire subsequent path;
\item {\bfseries eventually X} - X eventually has to hold, somewhere on the subsequent path;
\item {\bfseries after X, Y} - after the event X is observed, Y has to hold on the entire subsequent path;
\item {\bfseries until X, Y} - X holds at the current or future position, and Y has to hold until that position. At that position, Y does not have to hold anymore;
\item {\bfseries after\_until X, Y, Z} - after the event X is observed, Z has to hold on the entire subsequent path up until Y happens, at that position Z does not have to hold anymore;
\end{itemize}

%\noindent It is also possible to reference previous states of variables, using \textbf{\lstinline|@{X, -y}|}, representing the value of variable \lstinline|X| at time \lstinline|-y|.

% ------------------------------------------------------------------------------
% Temporal value
\subsection{Temporal value}

It is also possible to reference previous variable states:
\begin{equation}
@\{X, -y\}
\end{equation}
This will represent the value of the variable X in the point in time -y.

% ------------------------------------------------------------------------------
% Operands
\section{Operands}

Besides Number / Boolean / Var

\begin{itemize}
\item \textbf{Temporal value -} aa
\item \textbf{Function -} aa
\item \textbf{Property -} aa
\end{itemize}

% ------------------------------------------------------------------------------
% Functions
\subsection{Simulation primivitives}

\begin{itemize}
\item {\bfseries X.position} - The position of the robot in the simulation;
\item {\bfseries X.position.y} - The position in the y axis of the robot in the simulation. Also works for x and z;
\item {\bfseries X.distance.Y} - The absolute distance between two objects in the simulation. For the x and y axis;
\item {\bfseries X.distanceZ.Y} - The absolute distance between two objects in the simulation. For the x, y, and z axis;
\item {\bfseries X.velocity} - The velocity of an object in the simulation. This refers to linear velocity;
\item {\bfseries X.velocity.x} - The velocity in the x axis of an object in the simulation. This refers to linear velocity;
\item {\bfseries X.localization\_error} - The difference between the robot's perception of its position and the actual position in the simulation;
\end{itemize}

% ------------------------------------------------------------------------------
% Protected Variables
\section{Protected Variables}

\_rate\_ - Set the frame rate which properties are checked (By default the rate is 30hz)

\_timeout\_ - Set the timeout for how long the verification will last (By default the timeout is 100 seconds)

\_margin\_ - Set the error margin for comparisons

% ------------------------------------------------------------------------------
% Topic declaration
\section{Topic declaration}

In order to relate robot components with the simulation, the developer can declare the relevant \textit{topic}.

\textit{The variable robot\_position was declared with the type Odometry.pose.pose.position and is linked to the topic /odom;}

\vspace{2mm}

\texttt{decl robot\_position /odom Odometry.pose.pose.position}

% ------------------------------------------------------------------------------
% Model robots
\section{Model robots}

There are a set of specific topics that can be modeled for the robot, like \textit{position} or \textit{velocity}. These will be used by the compiler to call specific functions that need this information from the robot's perspective.

\vspace{2mm}

\texttt{model robot1:}

\texttt{    position /odom Odometry.pose.pose.position}

\texttt{    ;}

\vspace{2mm}

\texttt{never robot1.localization error > 0.002}

% ------------------------------------------------------------------------------
% Grammar
\section{Grammar}

\begin{bnfgrammar}
    program : Start
    ::=
    command
    | command program
    ;;
    command ::=
    association
    | declaration
    | model
    | pattern
    ;; 
    association ::=
    name = pattern
    | \_rate\_ = integer
    | \_timeout\_ = number
    | \_default\_margin\_ = number
\end{bnfgrammar}


   <declaration> → decl name topic\_name <msgtype>
                 | decl name name <msgtype>

         <model> → model name : <modelargs> ;

     <modelargs> → <name> topic\_name <msgtype>
                 | <name> <name> <msgtype>
                 | <name> topic\_name <msgtype> <modelargs>
                 | <name> <name> <msgtype> <modelargs>

          <name> → name
                 | <func\_main>

     <func\_main> → position
                 | velocity
                 | distance
                 | localization\_error
                 | orientation

       <msgtype> → <name>
                 | <name> . <msgtype>

       <pattern> → always <pattern>
                 | never <pattern>
                 | eventually <pattern>
                 | after <pattern> , <pattern>
                 | until <pattern> , <pattern>
                 | after\_until <pattern> , <pattern> , <pattern>
                 | <conjunction>

   <conjunction> → <conjunction> and <comparison>
                 | <conjunction> or <comparison>
                 | <conjunction> implies <comparison>
                 | <comparison>

    <comparison> → <multiplication> <opbin> <multiplication>
                 | <multiplication> <opbin> { <number> } <multiplication>
                 | <multiplication>

         <opbin> → <
                 | >
                 | <=
                 | >=
                 | ==
                 | !=

<multiplication> → <multiplication> * <addition>
                 | <multiplication> / <addition>
                 | <addition>

      <addition> → <addition> + <operand>
                 | <addition> - <operand>
                 | <operand>

       <operand> → name
                 | <number>
                 | true
                 | false
                 | <func>
                 | <temporalvalue>
                 | ( <pattern> )

        <number> → float
                 | integer

          <func> → name . <func\_main>
                 | name . <func\_main> <funcargs>

      <funcargs> → . <name>
                 | . <name> <funcargs>

 <temporalvalue> → @ { name , integer }
