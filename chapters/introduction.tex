% !TeX root = ../main.tex
% %%%%%%%%%%%%%%%%%%%%%%%%%%%%%%%%%%%%%%%%%%%%%%%%%%%%%%%%%%%%%%%%%%%%%%%%%%%%%%
% Introduction
\chapter{Introduction}
\label{chap:introduction}

This thesis aims at exploring a possible solution for automation in the testing of robotic systems through the medium of a domain-specific language and simulation software.

The intent of this chapter is to introduce the motivation for this work (Section 1.1), present the problem statements of such an approach (Section 1.2), discuss the objectives (Section 1.3), present the expected contributions (Section 1.4), and finally summarize the structure of the rest of the document (Section 1.5).


% ------------------------------------------------------------------------------
% Motivation
\section{Motivation}
\label{sec:motivation}

Robotics already have a significant impact on our current society, industrially (medicine, agriculture, etc.) or leisurely (contests, personal use, etc.) and often take critical roles like the example of robot arms in car assembly lines or autonomous farms. The tendency is for robot usage to keep growing at a global level. 

Robotic Systems are non-deterministic, mainly because robots interact directly with the real world. Testing software in such environments is complex, as there are many variables that can change, and verifying the success of a task or movement may not be possible from the robot's perspective, and external monitoring may be required.

Current practices in testing robot software mainly involve field testing, simulation testing, and log checking and require a human to analyze the behavior of the robot to determine whether the behavior is correct. Due to their broad practicality, the quality of software running on robots should be extremely important to us. Robot software as well as the techniques used to test their quality are very field-specific and different from the techniques employed in traditional Software Engineering mainly because of their real-world interaction, this means automatic tests are barely used in robotics. Studying possible options for viable automation of tests in robotic systems could lead to an opening on its usage in both research and the industry. Also, allowing for multiple parallel executions of tests not depending on human monitoring could improve the quality of current and future robot software.


\section{Problem Statement}
\label{sec:problem}

The multiple challenges in robot testing have an influence on planning how to test a robot because there are tradeoffs among choices.

In simulation the developers can take advantage of real values of objects' attributes to compare with what the robot system perceives, using this alleyway it is possible to in a way surpass the need for human-in-the-loop testing. 
 
While simulation-based tests are a promising approach for automation there is still distrust in the precision and validity of the results. Simulation-based autonomous testing is barely used due to not only reliability but also factors like cost and complexity. This means that, despite being dangerous, sometimes expensive, or work-intensive, real-life robot testing or other methods are still the main choices. The resulting product is a lack of quality in the software across projects.

When developing a domain-specific language for simulation-based autonomous tests, problems like what components to monitor and how to express them arise. Having a domain-specific language to specify a robotic system's properties can be useful but there is a need to control its complexity and accessibility or else it can become a burden in the testing process.


% ------------------------------------------------------------------------------
% Objectives
\section{Objectives}
\label{sec:objectives}

The ultimate goal of this thesis is to remove the need for human-in-the-loop testing of robotic systems, through the study of a possible solution for automation in simulation-based tests.

This work aims to provide developers with a way to verify their robotic systems' temporal and positional properties automatically. We propose the introduction of a domain-specific language for developers to express their relevant properties. The given properties are compiled into monitors that can be used in simulation to ensure the correctness of the system. The language was designed from the point of view of the Robot Operating System (ROS)~\cite{quigley2009ros} developers and tries to abstract the underlying Linear Temporal Logic (LTL) system, allowing properties to reason about native ROS constructs, like \textit{topics}, \textit{messages} and simulation information. Thus, it is possible to express properties that relate the internal information of the system with the corresponding information in the simulator.

The language should allow describing a robotic system's properties in a simple and intuitive way, while at the same time still being able to express relevant temporal and positional arguments between robots components and objects in the simulation.


\section{Contributions}
\label{sec:contributions}

The expected contributions of this thesis are below enumerated.

\begin{enumerate}
    \item Definition of a domain-specific language to specify robotic systems' properties.
    \item Implementation of a compiler for the language that can generate software capable of monitoring relevant components while in a simulation.
    \item Evaluation of the expressive capabilities of the solution.
\end{enumerate}

% ------------------------------------------------------------------------------
% Structure of the document
\section{Structure of the document}
\label{sec:structure}

The document is organized as follows:

\begin{itemize}
    \item Chapter 2 - Background \& Related Work:
    \item Chapter 3 - Proposed Approach:
    \item Chapter 4 -
\end{itemize}
