% !TeX root = ../main.tex
\chapter{Conclusion}
\label{chap:conclusion}


Due to the fact that robots interact with the real world, robotic systems are unpredictable. Coming up with a reliable and efficient method for automatic robot testing is a challenge, one of the reasons being that verifying the success of a task may not be possible from the robot's perspective. 

My approach takes advantage of simulation software to perform a type of external monitoring in order to achieve automatic monitoring of the robotic system. 

My approach relies on simulation-based testing so that developers can take advantage of the real values of objects' attributes in the simulation to compare with what the robot system perceives, trying in this way to surpass the need for human-in-the-loop testing.

I succeeded in developing a DSL that allows for the specification of a ROS robotic system's properties and that abstracts an underlying LTL system. It is possible to express relevant temporal and positional arguments between robots' components and objects in the simulation, and also properties that relate the internal information of the system with the corresponding information in the Gazebo simulator.

Although better validation by developers on the expressiveness of the DSL is still needed, I believe the developed DSL provides an easy-going, intuitive and in-depth way for both experts and non-experts to specify properties for robotic systems.

I have also succeeded in developing a tool for the generation of automatic monitoring software that can monitor some behavioral violations of robots in relation to their state or events during a Gazebo simulation. At the same time, I have shown that the approach's generated monitoring software can monitor some interesting scenarios that developers care about. 

%The developed work needs 
%I also present what is still left to do and what future work is still needed.
