% !TeX root = ../main.tex
% %%%%%%%%%%%%%%%%%%%%%%%%%%%%%%%%%%%%%%%%%%%%%%%%%%%%%%%%%%%%%%%%%%%%%%%%%%%%%%
% Resumo em Português
\selectlanguage{portuguese}

\vspace*{2cm}
\begin{center} \Large \bf Resumo
\end{center}
\vspace*{1cm} \setlength{\baselineskip}{0.6cm}

A Robótica tem uma grande influência na sociedade atual, ao ponto que a falha em algum sistema robótico que seja crucial pode impactar o modo em como nós vivemos, se, por exemplo, um carro autónomo provocar a morte de algum passageiro devido a um defeito, futuros e atuais utilizadores deste modelo irão certamente ficar apreensivos em relação à sua utilização. Assegurar que robôs reproduzam um comportamento correto pode assim salvar bastante dinheiro em estragos ou até mesmo as nossas vidas.

As práticas atuais em relação a testes de sistemas robóticos são vastas e envolvem métodos como simulações, verificação de “logs”, ou testagem em campo, frequentemente, um denominador comum entre estas práticas é a necessidade de um humano pessoalmente analisar e determinar se o comportamento de um sistema robótico é o correto. A automatização deste tipo de análise poderia não só aliviar o trabalho de técnicos especializados, facilitando assim a realização de testes, mas também possivelmente permitir a execução massiva de testes em paralelo que podem potencialmente detetar falhas no comportamento do sistema robótico que de outra maneira não seriam identificados devido a erros humanos ou à falta de tempo.

Apesar de existir alguma literatura relacionada com esta investigação, de uma maneira geral a automatização no campo da deteção de erros ou criação de invariantes continua a não ser adotada, pelo que o estudo apresentado nesta tese é justificado. 

Esta dissertação visa assim explorar o problema da automatização na deteção de erros comportamentais em robôs num ambiente de simulação, introduzindo uma linguagem de domínio específico direcionada a especificar as propriedades de sistemas robóticos em relação ao seu ambiente, assim como a geração de “software” de monitorização capaz de detetar a transgressão destas propriedades.

A linguagem de domínio específico necessita de expressar requisitos de determinados estados ou eventos durante a simulação, desta maneira precisa de apresentar determinadas características. Palavras-chave para representar relações temporais de ou entre objetos, como, por exemplo, o robô "nunca", ou "eventualmente" o robô. Referências a estados anteriores da simulação, como, por exemplo, a velocidade do robô está sempre a aumentar, ou seja, é sempre maior que no estado anterior. Atalhos para ser possível referir certas características de ou entre objetos, como, por exemplo, a "posição", "velocidade" ou "distância" de ou entre robôs.

A linguagem de domínio específico também assume que o sistema robótico irá ser executado por meio da framework ROS (Robot Operating System), que é amplamente utilizada para investigação e na indústria da robótica. A arquitetura do ROS engloba características como “publish-subscribe” entre "tópicos" e tipos de mensagem, estas características são tidas em conta e foram integradas no desenvolvimento da linguagem.

O “software” de monitorização gerado refere-se a um ficheiro python que correrá sobre a framework ROS. A geração deste ficheiro assume também que a monitorização será feita no simulador Gazebo, isto porque para obter dados como a posição ou velocidade absoluta de um robô durante a simulação é necessário aceder a "tópicos" ROS específicos que na geração do ficheiro de monitorização estão “hardcoded”. A geração de um ficheiro capaz de executar a monitorização significa que esta pode ser executada independente de um sistema robótico, permitindo assim a automatização da monitorização a respeito de vários objetos e as suas relações.

Resultados mostram que é possível expressar propriedades temporais e posicionais de e entre robôs e o seu ambiente com o suporte da linguagem de domínio específico. O trabalho mostra também que é possível automatizar a monitorização da violação de alguns tipos de comportamentos esperados de robôs em relação ao seu estado ou determinados eventos que ocorrem durante uma simulação.

*Evaluation ?*

*possiveis problemas, e futuro* - proof of language or that it works - better information on the errors - frequency of checking the properties can be modified in some circunstances to not check at every iteration - alargar a outros simuladores - integration with other tools liek scenario generation

%Para documentos em Português: Resumo em português até \textbf{300} palavras.
%Para documentos em língua estrangeira: Resumo em português entre \textbf{1200} e \textbf{1500} palavras.

\vfill

\begin{flushleft}
\textbf{Palavras-chave:}
Robótica, Linguagem de domínio, Deteção de erros, Automação, Monitorização
\end{flushleft}

\selectlanguage{english}