% !TeX root = ../main.tex
% %%%%%%%%%%%%%%%%%%%%%%%%%%%%%%%%%%%%%%%%%%%%%%%%%%%%%%%%%%%%%%%%%%%%%%%%%%%%%%
% Resumo em Português
\selectlanguage{portuguese}

\vspace*{2cm}
\begin{center} \Large \bf Resumo
\end{center}
\vspace*{1cm} \setlength{\baselineskip}{0.6cm}

A Robótica tem uma grande influência na sociedade atual, seja industrialmente, na medicina, na agricultura, ou como forma de lazer, e, muitas vezes, toma um papel crucial em que a falha em algum destes sistemas robóticos cruciais pode impactar o modo em como nós vivemos. Se, por exemplo, um carro autónomo provocar a morte de algum passageiro devido a um defeito, futuros e atuais utilizadores deste modelo irão certamente ficar apreensivos em relação à sua utilização. Assegurar que robôs reproduzam um comportamento correto pode assim salvar bastante dinheiro em estragos ou até mesmo as nossas vidas.

Os sistemas robóticos são não deterministas, isto porque os robôs interagem diretamente com o mundo real. Testar \textit{software} em um ambiente destes é bastante complexo devido a as variáveis serem imprevisíveis e mudarem constantemente. Verificar o sucesso de um movimento ou tarefa neste ambiente pode ser bastante difícil do ponto de vista de um robô, pelo que monitorização externa é muitas vezes necessária. 

Devido à ampla utilização de sistemas robóticos, a qualidade do software que corre em robôs deveria ser de extrema importância para nós. O \textit{software} destes sistemas, assim como os métodos utilizados para os testar, são bastante específicos da área e diferentes de \textit{software} tradicional, em grande parte devido à sua já falada interação com o mundo real.

As práticas atuais em relação à testagem de sistemas robóticos são vastas e envolvem métodos como simulações, verificação de “logs”, testes em campo, entre outras. Frequentemente, um denominador comum entre as práticas adotadas é a necessidade de um humano pessoalmente analisar e determinar se o comportamento de um sistema robótico é o correto. A automatização deste tipo de análise poderia não só aliviar o trabalho de técnicos especializados, facilitando assim a realização de testes, mas também possivelmente permitir a execução massiva de testes em paralelo. Este tipo de testagem automática poderá potencialmente detetar falhas no comportamento do sistema robótico que de outra maneira não seriam identificadas devido a erros humanos ou mesmo à falta de tempo.

Apesar de existir algum trabalho e literatura relacionada com a utilização de testes automáticos em sistemas robóticos, assim como ferramentas para realizar de alguma maneira este tipo de análise, de uma forma geral, a automatização no campo da deteção de erros ou até mesmo na utilização de invariantes continua a não ser adotada para este tipo de sistemas, isto devido à complexidade ou à falta de confiança nas soluções já desenvolvidas, o que incentiva o estudo apresentado nesta tese.

Uma invariante representa uma propriedade que se mantém durante toda a execução do sistema, dispor de uma lista de invariantes para um sistema robótico e ser capaz de as verificar em tempo de execução é uma forma de provar a qualidade desse sistema.

Esta dissertação visa assim explorar o problema da automatização da deteção de erros comportamentais em robôs num ambiente de simulação, introduzindo uma linguagem de domínio específico direcionada a especificar as propriedades de sistemas robóticos em relação ao seu ambiente, assim como a geração de \textit{software} de monitorização capaz de detetar a transgressão destas propriedades durante uma simulação.

A linguagem de domínio específico também assume que o sistema robótico irá ser executado por meio da framework de código aberto ROS (Robot Operating System). O ROS é uma framework que oferece uma vasta coleção de livrarias, interfaces e ferramentas especificamente desenhadas para ajudar na construção de \textit{software} para robôs. O ROS fornece uma abstração entre \textit{software} e hardware que ajuda desenvolvedores a facilmente conectar diferentes componentes de robôs através de mensagens enviadas por canais de comunicação chamados \textit{tópicos}. O ROS tem uma arquitetura modular e outras vantagens que têm como objectivo a intra-colaboração e fácil desenvolvimento de \textit{software}. O seu ecossistema está construído de maneira a que a maioria dos projetos dependem de uma pequena lista de pacotes. Devido a todos os factos em cima mencionadas o ROS é amplamente utilizado para investigação e na indústria da robótica, e por essas mesmas razões o escolhemos como a framework que seria integrada neste trabalho.

Lógica temporal linear pode ser usada como um método de verificação de programas e também ser útil para a criação de invariantes em sistemas robóticos. Um sistema formal de lógica temporal contém padrões que podem ser usados como uma forma de especificação de propriedades deste tipo de sistemas.

Lógica temporal linear é um ramo da lógica responsável por representar e relacionar componentes em referência a uma linha temporal. A linguagem de domínio específico necessita de expressar requisitos de determinados estados ou eventos durante a simulação, desta maneira precisa de apresentar determinadas características: Palavras-chave para representar relações temporais de ou entre objetos, como, por exemplo, o robô "nunca", ou "eventualmente" o robô. Referências a estados anteriores da simulação, como, por exemplo, a velocidade do robô está sempre a aumentar, ou seja, é sempre maior que no estado anterior. Atalhos para ser possível referir certas características de ou entre objetos, como, por exemplo, a "posição", "velocidade" ou "distância" de ou entre robôs.

O \textit{software} de monitorização gerado refere-se a um ficheiro \textit{python} que tem origem numa especificação feita através da linguagem de domínio específico e que correrá sobre a framework ROS. A geração deste ficheiro assume também que a monitorização será feita no simulador Gazebo, isto porque para obter dados como a posição ou velocidade absoluta de um robô durante a simulação é necessário aceder a \textit{tópicos} ROS específicos que na geração do ficheiro de monitorização estão \textit{hardcoded}, pois dependendem do \textit{software} de monitorização escolhido. 

O Gazebo começou com a ideia de um simulador de alta-fidelidade capaz de simular sistemas robóticos em qualquer tipo de ambiente ou condições. O Gazebo é um simulador de código aberto que suporta ferramentas como a simulação de sensores, manipulação de modelos, e o controlo de atuadores sob diferentes motores de física, entre outras ferramentas, o que o faz um simulador que vários sistemas robóticos diferentes entre si são capazes de tirar proveito, daí a sua escolha para o desenvolvimento deste trabalho.

A geração de um ficheiro capaz de executar a monitorização durante uma simulação significa também que este tipo de monitorização pode ser executada independente de um sistema robótico, permitindo assim a automatização da monitorização a respeito de vários objetos e as suas relações.

O objetivo deste trabalho é então fornecer uma maneira de desenvolvedores de \textit{software} para sistemas robóticos de conseguirem verificar propriedades posicionais e temporais dos seus sistemas de maneira automática através de uma linguagem de domínio específico, que deve, ao mesmo tempo, ser simples, intuitiva e permitir expressar propriedades que sejam relevantes entre componentes da simulação.

De maneira a avaliar o trabalho desenvolvido tentamos detetar erros em sistemas robóticos, inserindo propositadamente \textit{bugs} no sistema. Estes \textit{bugs} provêm de uma lista de \textit{bugs} que foram previamente identificados por outros utilizadores destes sistemas robóticos. A lista contém \textit{bugs} bastante diferentes entre si, dos quais a maior parte já foi corrigido em \textit{software} atual, sendo que o nosso objetivo é identificar \textit{bugs} antigos que ocorram em tempo de execução.

Resultados mostram que é possível expressar propriedades temporais e posicionais de e entre robôs e o seu ambiente com o suporte da linguagem de domínio específico. O trabalho mostra também que é possível automatizar a monitorização da violação de alguns tipos de comportamentos esperados de robôs em relação ao seu estado ou determinados eventos que ocorrem durante uma simulação.


%Para documentos em Português: Resumo em português até \textbf{300} palavras.
%Para documentos em língua estrangeira: Resumo em português entre \textbf{1200} e \textbf{1500} palavras.

\vfill

\begin{flushleft}
\textbf{Palavras-chave:}
Robótica, Linguagem de domínio, Monitorização em tempo de execução, Deteção de erros
\end{flushleft}

\selectlanguage{english}