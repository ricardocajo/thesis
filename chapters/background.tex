% !TeX root = ../main.tex
\chapter{Background \& Related Work}
\label{chap:background}

In this chapter as a background to the work, 
the Robot Operating System (ROS) is intruduced.
Has for related work, Gazebo is intruduced as the simulation software to be used 
and GzScenic as a language that allows to specify testing scenarios.

\par

ROS is an open-source framework with a vast collection 
of libraries and tools that help build robot software. 
ROS runs on Linux Ubuntu and provides an abstraction between hardware and software.
ROS was built with the purpose of cross-collaboration, there are
packages for almost everything and no need to reinvent the wheel.
ROS is the most widely used tool for writing robot software.
Companies like Sony, LG, Rapyuta Robotics, etc., rely on ROS to deliver their products~\cite{Ros}.

\par

Robot simulation is an essential tool for testing robots behavior. 
Gazebo started with the idea of a high-fidelity simulator to simulate 
robots in outdoor environments under varied conditions.
Today it offers the ability to simulate numerous robots in complex distinct environments.
Gazebo is an open-source 3D simulator that supports sensors simulation 
and actuators control under different physics engines~\cite{Gazebo}.

\par

Scenic is a domain-specific language used to describe scenarios.
It is a probabilistic programming language and it helps design the 
cyber-physical systems used in a simulation.
Scenic big potential is allowing the randomization of specific scenarios delimited by the language.
Although this is true Scenic as a problem, it is designed to support only 
a specific number of simulators, mainly vehicle simulation~\cite{Scenic}.

\par

With this in mind, GzScenic was built. The GzScenic tool allows the generation of this type of 
scenarios in the Gazebo simulator, which is the most popular general-purpose simulator~\cite{GzScenic}.