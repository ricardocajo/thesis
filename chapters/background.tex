% !TeX root = ../main.tex
\chapter{Background \& Related Work}
\label{chap:background}

This chapter gives an overview of the software adopted while developing this work (section 2.1), followed by a brief explanation of Linear Temporal Logic (section 2.2), and finally shines some light on the already existing similar work on the subject (section 2.3).

\section{Software}

This section provides some background on the used software and the reason for its choice, what the Robot Operating System is (subsection 2.1.1), and the simulation software adopted (subsection 2.1.2).

% ------------------------------------------------------------------------------
% ROS
\subsection{Robot Operating System}
\label{sec:ros}

The Robot Operating System (ROS)~\cite{quigley2009ros} is an open-source framework with a vast collection of libraries, interfaces, and tools designed to help build robot software. ROS provides an abstraction between hardware and software that helps developers easily connect the different robot components through messages sent through communication channels (\textit{topics}).

ROS has a modular architecture and other advantages built with the purpose of cross-collaboration and easy development. For all these reasons, ROS is used by hundreds of companies and research labs.

% ------------------------------------------------------------------------------
% Gazebo
\subsection{Gazebo}
\label{sec:gazebo}

Robotic systems simulation is an essential tool for testing robots' behavior. For this reason, Gazebo~\cite{koenig2004design} started with the idea of a high-fidelity simulator to simulate robots in any environment under mixed conditions.

Gazebo is an open-source 3D simulator that supports tools like sensors simulation, mesh management, and actuators control under different physics engines, among others, which makes it a simulator that very distinct robotic systems can use.

% ------------------------------------------------------------------------------
% Linear Temporal Logic
\section{Linear Temporal Logic}
\label{sec:ltl}

Linear temporal logic (LTL) is a branch of logic responsible for representing and reasoning about modalities in reference to time. 

As an approach for program verification, a formal system of temporal logic was suggested for both sequential and parallel programs~\cite{pnueli1977temporal}. LTL can be used as a method of model-checking~\cite{dwyer1998property} using its patterns as a form of property specification. It includes patterns such as "always", "finally", "until", "eventually", and others, which can be useful in the creation of invariants for program verification.

% ------------------------------------------------------------------------------
% Property Specification
\section{Robot Testing}
\label{sec:robottesting}

Some research on the importance of invariants checking (subsection 2.3.1) and runtime testing and the difficulties of implementing it already exist (subsection 2.3.2). As well as some tools that already try to implement similar runtime verification concepts (subsection 2.3.3).

\subsection{Invariants}

An invariant represents a property that holds through the execution of the system. Having a set of invariants for a robotic system and asserting them at runtime makes it able to prove the correctness of the system.

Research on invariant checking~\cite{zizyte2021importance} shows that a considerable amount of bugs on autonomous robotic systems can be avoided when representing safety violations of systems and monitoring them.

\subsection{Runtime Monitoring}

Due to the unforeseen circumstances mentioned when executing robotic systems, runtime monitoring, although sometimes time-consuming, may be advantageous when identifying errors in these types of systems.

Implementing runtime monitoring adds load to the simulation. Therefore, not demanding excessive resources is essential when taking this approach.

Some challenges in implementing such mechanisms are mentioned in the cited paper~\cite{stadler2022towards}.

\subsection{Similar work}

Similar work on runtime monitoring that integrates with ROS already exists. 

ROSMonitoring~\cite{ferrando2020rosmonitoring} can monitor and log errors at the level of \textit{topic} malfunctioning, but it seems unable to express more high-level properties, which is the objective of this work.

ROSRV~\cite{huang2014rosrv} although able to express more high-level specifications, it is highly complex and, in some way, hard for non-expert users to work with. An intuitive domain-specific language will allow a broader set of users to specify a robotic system's properties.
