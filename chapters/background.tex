% !TeX root = ../main.tex
\chapter{Background \& Related Work}
\label{chap:background}

In order to to have a better understanding of the developed work some background on important concepts is given, in particular the Robot Operating System (ROS)~\cite{Ros}, Gazebo~\cite{Gazebo}, and literal temporal logic.

Some understanding on the current state of the art is also needed,

% ------------------------------------------------------------------------------
% ROS
\section{Robot Operating System}
\label{sec:ros}

ROS is an open-source framework with a vast collection of libraries, interfaces, and tools that help build robot software. 

ROS provides an abstraction between hardware and software that helps developers easily connect the different robot components throught what is called "topics" and "messages".

ROS has a modular architecture along other advantages that were built with the purpose of cross-collaboration and easy development. For all these reasons ROS is used by hundreds of companies and research labs.

% ------------------------------------------------------------------------------
% Gazebo
\section{Gazebo}
\label{sec:gazebo}

Robotic systems simulation is an essential tool for testing robots behavior, for this reason Gazebo started with the idea of a high-fidelity simulator to simulate robots in any type of environment under mixed conditions.

Gazebo is an open-source 3D simulator that supports tools like sensors simulation, mesh management, actuators control under different physics engines, among others, which makes it a simulator that is used by very distinct robotic systems.

% ------------------------------------------------------------------------------
% Linear Temporal Logic
\section{Linear Temporal Logic}
\label{sec:ltl}



% ------------------------------------------------------------------------------
% Property Specification
\section{Robot Testing}
\label{sec:robottesting}

\subsection{Challenges}

\subsection{Invariants}

% ------------------------------------------------------------------------------
% Property Specification
\section{Property Specification}
\label{sec:propspecification}
